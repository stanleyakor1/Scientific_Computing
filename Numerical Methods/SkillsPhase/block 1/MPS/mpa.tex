\documentclass[10pt,a4paper]{article}
\usepackage[utf8]{inputenc}
\usepackage{amsmath}
\usepackage{amsfonts}
\usepackage{amssymb}
\newtheorem{theorem}{Theorem}[section]
\newtheorem{proof}[theorem]{proof}
\author{Stanley}
\title{mps}
\begin{document}
\maketitle

\section{theorem}
\section{Theorem}
\begin{theorem}
For any triangulated polygon, we can assign three colours to the vertex, so that all the vertex of any polygon may be expressed in three colours
\end{theorem}

\begin{proof} 
\end{proof}
Base Case: \\
Considering the very trivial case of a simple polygon [that is a triangle] where $n=3$, we actually see that for each vertex we can assign a different colour.\\
\\
The colouratiaon principle  can then be  extended for larger polygons $n=4,5,.........,k$ and this holds true as shown in the figure below.




\section{Conclusion}
The minimum number of guards required by brian to secure the entire perimeter of his farm is 7 as indicated by the red colours of the 24-sided polygons. 

The number of guards required to cover any n-gon shaped field is proportional to the number of sides of the polygon, the solution is found by taking the floor of $\frac{n}{3}$ where $n$ is the number of sides of the n-gon, however this value is not often the minimum number of guards required.


\end{document}