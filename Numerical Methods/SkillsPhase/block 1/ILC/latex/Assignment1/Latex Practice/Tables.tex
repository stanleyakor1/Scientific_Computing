\documentclass[10pt,a4paper]{article}
\usepackage[utf8]{inputenc}
\usepackage{amsmath}
\usepackage{amsfonts}
\usepackage{amssymb}
\author{Stanley}
\title{tables}
\begin{document}

\begin{table}[h!]
\centering
\begin{tabular}{r l}
\hline
8 & here\\
\cline{2-2}
86 & stuff\\
\hline
\hline
2008 & now\\
\hline

\end{tabular}
\caption{ Table Name}

\end{table}

\begin{enumerate}
\item Students
\begin{itemize}
\item Stanley
\item Patricia
\item Olive
\end{itemize}
\item Tutors
\begin{itemize}
\item Phil
\item Hove
\end{itemize}
\end{enumerate}

\section{Course work}
As the world is keen on reducing the constant threat posed by climate change, accurate short-term forecasts
will aid.

As the world is keen on reducing the constant threat posed by climate change, accurate short-term forecasts
will aid.

\subsection{ Recap} \label{Recap}
As the world is keen on reducing the constant threat posed by climate change, accurate short-term forecasts
will aid.

\subsection{Turbulence Intensity}
Turbulence in wind is associated to many factors such as, temperature, pressure, atmospheric stability, e.t.c. Several methods have been developed to assess turbulence, the stochastic method  uses the standard deviation of the vertical mean wind speed over 100m. Mathematically, it is represented as:
\begin{align}
\tau=\frac{\sigma_{u}}{\bar{u}}
\end{align} 
Where $\tau$ is the turbulence intensity, $\bar{u}$ is the mean wind speed and $\sigma_{u}$ is the standard deviation of the variation in wind speed.


\begin{align}
R^{2}=1-\frac{\sum_{i=1}^{N}(y_{i}-x_{i})^{2}}{\sum_{i=1}^{N}(y_{i}-\bar{y_{i}})^{2}}
\end{align}



\end{document}

