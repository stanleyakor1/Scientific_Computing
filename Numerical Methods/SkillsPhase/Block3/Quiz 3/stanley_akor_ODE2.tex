%%%%%%%%%%%%%%%%%%%%%%%%%%%%%%%%%%%%%%%%%%%%%%%%%%%%%%%%%%%%%%%%%%%%%%%%%%%%%%%%
%%%%%%%%%%%%%%%%%%%%%%%%%%%%%%%%%%%%%%%%%%%%%%%%%%%%%%%%%%%%%%%%%%%%%%%%%%%%%%%%
%%% Template for AIMS Rwanda Assignments         %%%              %%%
%%% Author:   AIMS Rwanda tutors                             %%%   ###        %%%
%%% Email: tutors2017-18@aims.ac.rw                               %%%   ###        %%%
%%% Copyright: This template was designed to be used for    %%% #######      %%%
%%% the assignments at AIMS Rwanda during the academic year %%%   ###        %%%
%%% 2017-2018.                                              %%%   #########  %%%
%%% You are free to alter any part of this document for     %%%   ###   ###  %%%
%%% yourself and for distribution.                          %%%   ###   ###  %%%
%%%                                                         %%%              %%%
%%%%%%%%%%%%%%%%%%%%%%%%%%%%%%%%%%%%%%%%%%%%%%%%%%%%%%%%%%%%%%%%%%%%%%%%%%%%%%%%
%%%%%%%%%%%%%%%%%%%%%%%%%%%%%%%%%%%%%%%%%%%%%%%%%%%%%%%%%%%%%%%%%%%%%%%%%%%%%%%%


%%%%%% Ensure that you do not write the questions before each of the solutions because it is not necessary. %%%%%% 

\documentclass[12pt,a4paper]{article}

%%%%%%%%%%%%%%%%%%%%%%%%% packages %%%%%%%%%%%%%%%%%%%%%%%%
\usepackage{amsmath}
\usepackage{amssymb}
\usepackage{amsthm}
\usepackage{amsfonts}
\usepackage{graphicx}
\usepackage[all]{xy}
\usepackage{tikz}
\usepackage{verbatim}
\usepackage[left=2cm,right=2cm,top=3cm,bottom=2.5cm]{geometry}
\usepackage{hyperref}
\usepackage{caption}
\usepackage{subcaption}
\usepackage{psfrag}

%%%%%%%%%%%%%%%%%%%%% students data %%%%%%%%%%%%%%%%%%%%%%%%
\newcommand{\student}{Akor stanley}
\newcommand{\course}{ODE }
\newcommand{\assignment}{2}

%%%%%%%%%%%%%%%%%%% using theorem style %%%%%%%%%%%%%%%%%%%%
\newtheorem{thm}{Theorem}
\newtheorem{lem}[thm]{Lemma}
\newtheorem{defn}[thm]{Definition}
\newtheorem{exa}[thm]{Example}
\newtheorem{rem}[thm]{Remark}
\newtheorem{coro}[thm]{Corollary}
\newtheorem{quest}{Question}[section]

%%%%%%%%%%%%%%  Shortcut for usual set of numbers  %%%%%%%%%%%

\newcommand{\N}{\mathbb{N}}
\newcommand{\Z}{\mathbb{Z}}
\newcommand{\Q}{\mathbb{Q}}
\newcommand{\R}{\mathbb{R}}
\newcommand{\C}{\mathbb{C}}

%%%%%%%%%%%%%%%%%%%%%%%%%%%%%%%%%%%%%%%%%%%%%%%%%%%%%%%555
\begin{document}

%%%%%%%%%%%%%%%%%%%%%%% title page %%%%%%%%%%%%%%%%%%%%%%%%%%
\thispagestyle{empty}
\begin{center}
\textbf{AFRICAN INSTITUTE FOR MATHEMATICAL SCIENCES \\[0.5cm]
(AIMS RWANDA, KIGALI)}
\vspace{1.0cm}
\end{center}

%%%%%%%%%%%%%%%%%%%%% assignment information %%%%%%%%%%%%%%%%
\noindent
\rule{17cm}{0.2cm}\\[0.3cm]
Name: \student \hfill Assignment Number: \assignment\\[0.1cm]
Course: \course \hfill Date: \today\\
\rule{17cm}{0.05cm}
\vspace{1.0cm}




\section*{Question 1}
\begin{itemize}
\item[(i)]
$\frac{dx}{dt}=|x|^{\frac{1}{2}}$\\
Case 1:$-x^{\frac{1}{2}}$\\
Case 2:$x^{\frac{1}{2}}$
\begin{align*}
\frac{dx}{dt}= -x^{\frac{1}{2}}
\end{align*}

Applying seperation of variables techniques,

\begin{align*}
\frac{dx}{-x^{\frac{1}{2}}}&=dt\\
\int -x^{\frac{1}{2}} dx&=dt\\
\int -x^{\frac{1}{2}}dx&=t+c\\
x&=\sqrt{\frac{-t}{2}+c}
\end{align*}
Taking similar steps for case 2, we shall obtain the second solution as;
$x=\sqrt{\frac{t}{2}+c}$

\item[(ii)]
\begin{align*}
x\frac{dx}{dt}&=t\\
xdx&=tdt\\
\int xdx&=\int tdt\\
\frac{{x^{2}}}{2}&=\frac{t^{2}}{2}+c\\
x&=\pm \sqrt{t^{2}+c}
\end{align*}

\item[(2)]
\begin{itemize}
\item[(i)] 
If a function $f(z)$ and its partial derivatives are continuous in domain of $z$, $i.e$ the function is defined, then the hypothesis of Picard-Lindelof's theorem are satisfied \\
\begin{align*}
\frac{dx}{dt}=|x|^{\frac{1}{2}}
\end{align*}
Taking partial derivative of this fuction with respect to $x$ yields;\\

\begin{align*}
\frac{\partial f(x,t)}{\partial t}=\frac{1}{2\sqrt{x}}
\end{align*}
The result is not continuous for $x=0$, hence the hypothesis of Picard-Lindelof's theorem are not satisfied

\item[(ii)]
\begin{align*}
x\frac{dx}{dt}&=t\\
x^{'}=\frac{t}{x}
\end{align*}
The function is not continuous at $x=0$, hence the hypothesis of the Picard-Lindelof theorem are not satisfied.
\end{itemize}

\section*{Question 2}
$\frac{du}{dt}=Au$ \quad \quad$u\left(0\right)=u_{0}$
\begin{itemize}
\item[(i)]
\begin{equation*}
A = 
\begin{pmatrix}
-5 & -4 & 2 \\
-2 & -2 & 2 \\
4 & 2 & 2
\end{pmatrix}
\end{equation*}

\begin{equation*}
A = 
\begin{pmatrix}
-5-\lambda & -4 & 2 \\
-2 & -2-\lambda & 2 \\
4 & 2 & 2-\lambda
\end{pmatrix}
\end{equation*}
\[
\left(-5-\lambda\right)
\begin{bmatrix}
    -2-\lambda  &  2     \\
    2  &  2-\lambda      
\end{bmatrix} 
+4
\begin{bmatrix}
    -2  &  -4      \\
    4 &  2      
\end{bmatrix} 
+2
\begin{bmatrix}
    -2  &  -2-\lambda      \\
    4 &  2      
\end{bmatrix} 
\]

$\left((-5-\lambda\right)[\left(-2-\lambda\right)\left(-2-\lambda\right)-4]+4[-2\left(2\right)-\left(4\right)\left(-4\right)]+2[-2\left(2\right)-\left(4\right)\left(-2-\lambda\right)]$\\
$$\left(-5-\lambda\right)\left(\lambda^{2}-8\right)+4\left(-4+16\right)+2\left(-4+8+4\lambda\right)$$
$$-\lambda^{3}-5\lambda^{2}+24\lambda=0$$
$$\left(\lambda^{2}+5\lambda-24\right)-\lambda=0$$
$$\lambda_{1}=0 \quad \lambda_{2}=-8 \quad \lambda_{3}=3$$

The eigen vectors attributed to $\lambda_{1}=0 $ can be obtained as follows;\\

\[
\begin{pmatrix} -5 & -4 & 2 \\ -2 & -2 & 2\\4 &2 &2 \end{pmatrix} \begin{pmatrix} x \\ y \\z \end{pmatrix}=\begin{pmatrix} 0 \\ 0 \\0 \end{pmatrix}
\]
\begin{align}
-5x-4y+2z=0\\
-2x-2y+2z=0\\
4x+2y+2z=0
\end{align}
Resolving equations(1),(2)and (3) simultaneously, we shall obtain the associated eigen vectors.\\
\begin{align*}
\lambda_{1}=0\quad \quad u_{1}=
\begin{pmatrix} -2 \\ 3 \\1 \end{pmatrix}
\end{align*}

Similarly we can obtain the eigen vectors for the eigen value $\lambda_{2}=3$.\\
\[
\begin{pmatrix} -5-\left(3\right) & -4 & 2 \\ -2 & -2-\left(3\right) & 2\\4 &2 &2-\left(3\right) \end{pmatrix} \begin{pmatrix} x \\ y \\z \end{pmatrix}=\begin{pmatrix} 0 \\ 0 \\0 \end{pmatrix}\]\\
\[
\begin{pmatrix} -8 & -4 & 2 \\ -2 & -5 & 2\\4 &-1 &10 \end{pmatrix} \begin{pmatrix} x \\ y \\z \end{pmatrix}=\begin{pmatrix} 0 \\ 0 \\0 \end{pmatrix}
\]\\
\begin{align}
-8x-4y+2z=0\\
-2x-5y+2z=0\\
4x+2y+-z=0
\end{align}
Resolving equations(4),(5)and (6) simultaneously, we shall obtain the associated eigen vectors.\\
\begin{align*}
\lambda_{2}=3\quad \quad u_{2}=
\begin{pmatrix} 1 \\ 6 \\16 \end{pmatrix}
\end{align*}

Similarly we can obtain the eigen vectors for the eigen value $\lambda_{3}=-8$.\\
\[
\begin{pmatrix} -5-\left(-8\right) & -4 & 2 \\ -2 & -2-\left(-8\right) & 2\\4 &2 &2-\left(-8\right) \end{pmatrix} \begin{pmatrix} x \\ y \\z \end{pmatrix}=\begin{pmatrix} 0 \\ 0 \\0 \end{pmatrix}\]\\
\[
\begin{pmatrix} 3 & -4 & 2 \\ -2 & 6 & 2\\4 &2 &10 \end{pmatrix} \begin{pmatrix} x \\ y \\z \end{pmatrix}=\begin{pmatrix} 0 \\ 0 \\0 \end{pmatrix}
\]\\
\begin{align}
3x-4y+2z=0\\
-2x+6y+2z=0\\
4x+2y+10z=0
\end{align}
Resolving equations(4),(5)and (6) simultaneously, we shall obtain the associated eigen vectors.\\
\begin{align*}
\lambda_{3}=-8\quad \quad u_{3}=
\begin{pmatrix} -2 \\ -1 \\1 \end{pmatrix}
\end{align*}
From the given equation;\\
\begin{align*}
\frac{du}{dt}&=Au\\
\frac{du}{u}&=Adt\\
\int \frac{du}{u}&=\int Adt\\
\ln |u|&=At+c\\
u&=ce^{tA}
\end{align*}
 Applying the initial value condition $u\left(0\right)=u_{0}$ we shall obtain;\\
\begin{align*}
u_{0}&=ce^{0}\\
c&=u_{0}
\end{align*}
$u_{1}(t)=e^{\lambda_{1}t}u_{1}=e^{0}\begin{pmatrix} -2 \\ 3 \\1 \end{pmatrix}$\\
$u_{2}(t)=e^{\lambda_{2}t}u_{1}=e^{3t}\begin{pmatrix} 1 \\ 6 \\16 \end{pmatrix}$\\
$u_{3}(t)=e^{\lambda_{3}t}u_{1}=e^{-8t}\begin{pmatrix} -2 \\ -1 \\1 \end{pmatrix}$\\
$u\left(t\right)=c_{1}e^{\lambda_{1} t}+c_{2}e^{\lambda_{2} t}+c_{3}e^{\lambda_{3} t}$\\
$u\left(t\right)=c_{1}\begin{pmatrix} -2 \\ 3 \\1 \end{pmatrix}+c_{2}e^{3t}\begin{pmatrix} 1 \\ 6 \\16 \end{pmatrix}+c_{3}e^{-8t}\begin{pmatrix} -2 \\ -1 \\1 \end{pmatrix}$\\
At $t=0 \quad u\left(0\right)=u_{0}$\\
\begin{align}
\begin{pmatrix} 2 \\ 1 \\2 \end{pmatrix}=
\begin{pmatrix} -2 & 1 &-2 \\ 3 & 6& -1\\1 &16&1 \end{pmatrix} \begin{pmatrix} c_{1} \\ c_{2}\\c_{3} \end{pmatrix}
\end{align}
\begin{align}
-2c_{1}+-c_{2}+-2c_{3}&=2\\
3c_{1}+6c_{2}+-c_{3}&=1\\
c_{1}+16c_{2}+c_{3}&=2
\end{align}
Resolving equations (11),(12)and (13) simultaneously, we shall obtain the value of the constants as;\\
$$c_{1}=\frac{-1}{4}\quad c_{2}=\frac{2}{11}\quad c_{3}=\frac{-29}{44}$$\\
The general solution is given as;\\
$u\left(t\right)=\frac{-1}{4}\begin{pmatrix} -2 \\ 3 \\1 \end{pmatrix}+\frac{2}{11}e^{3t}\begin{pmatrix} 1 \\ 6 \\16 \end{pmatrix}+\frac{-29}{44}e^{-8t}\begin{pmatrix} -2 \\ -1 \\1 \end{pmatrix}$\\
\item[(ii)]
\begin{equation*}
A = 
\begin{pmatrix}
4 & -3 & 0 \\
3 & 4 & 0 \\
5 & 10 & 10
\end{pmatrix}
\end{equation*}
We want to determine the eigen values of the matrix $A$.\\
\begin{equation*}
\begin{pmatrix}
4-\lambda & -3 & 0 \\
3 & 4-\lambda & 0 \\
5 & 10 & 10-\lambda
\end{pmatrix}
\end{equation*}
We want to obtain the obtain the determinant the of the matrix $A$.\\
$$\left(4-\lambda\right)\left(\left(4-\lambda\right)\left(10-\lambda\right)\right)+3\left(3\left(10-\lambda\right)\right)=0$$
$$-\lambda^{3}+18\lambda^{2}-105\lambda+250=0$$
$$\left(\lambda_{1}-10\right)\left(\lambda_{2}-4+3i\right)\left(\lambda_{3}-4-3i\right)$$
$$\lambda_{1}=10\quad \lambda_{2}=4-3i\quad\lambda_{3}=4+3i$$
Now we want to determine the eigen vectors associated with the eigen values;\\
$\lambda_{1}=10$
\[
\begin{pmatrix} 4-\left(10\right) & -3 & 0 \\  3& 4-\left(10\right) & 0\\5&10 &10-\left(10\right) \end{pmatrix} \begin{pmatrix} x \\ y \\z \end{pmatrix}=\begin{pmatrix} 0 \\ 0 \\0 \end{pmatrix}\]\\
\[
\begin{pmatrix} -6 & -3 & 0 \\ 3 & -6 & 0\\5 &10 &0 \end{pmatrix} \begin{pmatrix} x \\ y \\z \end{pmatrix}=\begin{pmatrix} 0 \\ 0 \\0 \end{pmatrix}\]
\begin{align}
-6x-3y+0=0\\
3x+6y+0=0\\
5x+10y+0=0
\end{align}
Resolving equations(14),(15)and (16) simultaneously, we shall obtain the associated eigen vectors.\\

\begin{align*}
\lambda_{1}=10\quad \quad u_{1}=
\begin{pmatrix} 0 \\ 0 \\1 \end{pmatrix}
\end{align*}
Similarly for $\lambda_{2}=4+3i$\\

\[
\begin{pmatrix} 4-\left(4+3i\right) & -3 & 0 \\  3& 4-\left(4+3i\right) & 0\\5&10 &10-\left(4+3i\right) \end{pmatrix} \begin{pmatrix} x \\ y \\z \end{pmatrix}=\begin{pmatrix} 0 \\ 0 \\0 \end{pmatrix}\]

By resolving the above matrix, we shall obtain;
\begin{align*}
\lambda_{2}=4+3i\quad \quad u_{1}=
\begin{pmatrix} 12-9i \\ -9+12i \\25 \end{pmatrix}
\end{align*}

The eigen vectors associated with the eigen value$\lambda_{3}=4-3i$ is the conjugate of the eigen vectors of  $\lambda_{2}=4+3i$\\
\begin{align*}
\lambda_{3}=4-3i\quad \quad u_{1}=
\begin{pmatrix} 12+9i \\ -9-12i \\25 \end{pmatrix}
\end{align*}

$u_{1}(t)=e^{\lambda_{1}t}u_{1}=e^{10t}\begin{pmatrix} 0 \\ 0 \\1 \end{pmatrix}$\\
$u_{2}(t)=e^{\lambda_{2}t}u_{1}=e^{4t}\left(\cos\left(3t\right)\begin{pmatrix} -12 \\ -9 \\25\end{pmatrix}-\sin\left(3t\right)\begin{pmatrix} -9 \\ 12 \\0\end{pmatrix}\right)$\\
$u_{3}(t)=e^{\lambda_{2}t}u_{1}=e^{4t}\left(\cos\left(3t\right)\begin{pmatrix} -9 \\ 12 \\0\end{pmatrix}+\sin\left(3t\right)\begin{pmatrix} -12 \\ -9 \\25\end{pmatrix}\right)$

$u(t)=c_{1}e^{\lambda_{1}t}u_{1}=e^{10t}\begin{pmatrix} 0 \\ 0 \\1 \end{pmatrix}+c_{2}e^{4t}\left(\cos\left(3t\right)\begin{pmatrix} -12 \\ -9 \\25\end{pmatrix}-\sin\left(3t\right)\begin{pmatrix} -9 \\ 12 \\0\end{pmatrix}\right)$ \\
$+c_{3}e^{4t}\left(\cos\left(3t\right)\begin{pmatrix} -9 \\ 12 \\0\end{pmatrix}+\sin\left(3t\right)\begin{pmatrix} -12 \\ -9 \\25\end{pmatrix}\right)$\\
At $t=0 \quad u\left(0\right)=u_{0}$

\begin{align}
\begin{pmatrix} 2 \\ 1 \\2 \end{pmatrix}=
\begin{pmatrix} 0 & -12 &-9 \\ 0 & -9& 12\\1 &25&0 \end{pmatrix} \begin{pmatrix} c_{1} \\ c_{2}\\c_{3} \end{pmatrix}
\end{align}
\begin{align}
-12c_{2}+-9c_{3}&=2\\
-9c_{2}+12c_{3}&=1\\
c_{1}+25c_{2}&=2
\end{align}
Resolving equations (18),(19)and (20) simultaneously, we shall obtain the value of the constants as;\\
$$c_{1}=\frac{17}{3}\quad c_{2}=\frac{-11}{75}\quad c_{3}=\frac{-2}{75}$$\\
Our general solution now becomes;\\
$u(t)=\frac{17}{3} e^{\lambda_{1}t}u_{1}=e^{10t}\begin{pmatrix} 0 \\ 0 \\1 \end{pmatrix}+\frac{-11}{75} e^{4t}\left(\cos\left(3t\right)\begin{pmatrix} -12 \\ -9 \\25\end{pmatrix}-\sin\left(3t\right)\begin{pmatrix} -9 \\ 12 \\0\end{pmatrix}\right)$ \\
$+\frac{-2}{75} e^{4t}\left(\cos\left(3t\right)\begin{pmatrix} -9 \\ 12 \\0\end{pmatrix}+\sin\left(3t\right)\begin{pmatrix} -12 \\ -9 \\25\end{pmatrix}\right)$\\
\end{itemize}

\section*{Question 3}
\item[(i)]
\begin{align*}
\frac{du}{dt}&=Au+F\left(u,t\right)   \quad \quad on \quad  [a,b]\\
e^{-At}\frac{du}{dt}&=e^{-At}\left(Au+F\left(u,t\right)\right)\\
\frac{du}{dt}\left(ue^{At}\right)&=e^{-At}F\left(u,t\right)\\
ue^{-At}|_{tn}^{t_{n+1}}&=\int\limits_{t_{n}}^{tn+1}e^{-At}F\left(u,t\right)\\
u(t_{n+1})e^{At_{n+1}}-u(t_{n})e^{At_{n}}&=\int\limits_{t_{n}}^{tn+1}e^{-At}F\left(u,t\right)\\
u(t_{n+1})e^{At_{n+1}}&=u(t_{n})e^{At_{n}}+\int\limits_{t_{n}}^{tn+1}e^{-At}F\left(u,t\right)\\
\end{align*}
Dividing through by $e^{A\left(t_{n+1}\right)}$
\begin{align*}
u(t_{n+1})&=u(t_{n})e^{At_{n}}e^{-At_{n+1}}+\int\limits_{t_{n}}^{tn+1}e^{At_{n+1}}e^{-At}F\left(u,t\right)\\
u(t_{n+1})&=u(t_{n})e^{A\left(-t_{n}+t_{n+1}\right)}+\int\limits_{t_{n}}^{tn+1}e^{A\left(t_{n+1}-t_{n}\right)}F\left(u,t\right)\\
\end{align*}
Setting $h=t_{n+1}-t_{n}$ , $t=t_{n}+\tau$ and $\tau=t-t_{n}$ we shall obtain the desired solution.\\

\begin{align*}
u(t_{n+1})&=u(t_{n}) e^{A(-t_{n}+t_{n+1})}+\int\limits_{0}^{h}e^{-(\tau-h)A}F(u(t_{n}+\tau),t_{n}+\tau)d\tau\\
\end{align*}
\newpage

\item[(ii)]
.\\
$F\left(u_{n},t_{n}\right)=F_{n}$
\begin{align*}
u(t_{n+1})&=u(t_{n})e^{A\left(-t_{n}+t_{n+1}\right)}+\int\limits_{0}^{h}e^{-\left(\tau-h\right)A}F\left(u\left(tn+\tau\right),t_{n}+\tau\right)d\tau\\
&=e^{hA}u_{n}+\int\limits_{0}^{h}e^{-\left(\tau-h\right)A}F(u(t_{n}+\tau),t_{n}+\tau)d\tau\\
&=u(t_{n})e^{hA}+\int\limits_{0}^{h}e^{-\left(\tau-h\right)A}F_{n}d\tau\\
&=u(t_{n})e^{hA}-A^{-1}F_{n}e^{-\left(\tau-h\right)}\lvert _{0}^{h}\\
&=u(t_{n})e^{hA}-A^{-1}F_{n}\left(I-e^{hA}\right)\\
&=e^{hA}u(t_{n})+A^{-1}\left(e^{hA}-I\right)F_{n}
\end{align*}
\end{itemize}
\end{document}