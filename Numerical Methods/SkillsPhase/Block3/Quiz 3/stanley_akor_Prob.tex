%%%%%%%%%%%%%%%%%%%%%%%%%%%%%%%%%%%%%%%%%%%%%%%%%%%%%%%%%%%%%%%%%%%%%%%%%%%%%%%%
%%%%%%%%%%%%%%%%%%%%%%%%%%%%%%%%%%%%%%%%%%%%%%%%%%%%%%%%%%%%%%%%%%%%%%%%%%%%%%%%
%%% Template for AIMS Rwanda Assignments         %%%              %%%
%%% Author:   AIMS Rwanda tutors                             %%%   ###        %%%
%%% Email: tutors2017-18@aims.ac.rw                               %%%   ###        %%%
%%% Copyright: This template was designed to be used for    %%% #######      %%%
%%% the assignments at AIMS Rwanda during the academic year %%%   ###        %%%
%%% 2017-2018.                                              %%%   #########  %%%
%%% You are free to alter any part of this document for     %%%   ###   ###  %%%
%%% yourself and for distribution.                          %%%   ###   ###  %%%
%%%                                                         %%%              %%%
%%%%%%%%%%%%%%%%%%%%%%%%%%%%%%%%%%%%%%%%%%%%%%%%%%%%%%%%%%%%%%%%%%%%%%%%%%%%%%%%
%%%%%%%%%%%%%%%%%%%%%%%%%%%%%%%%%%%%%%%%%%%%%%%%%%%%%%%%%%%%%%%%%%%%%%%%%%%%%%%%


%%%%%% Ensure that you do not write the questions before each of the solutions because it is not necessary. %%%%%% 

\documentclass[12pt,a4paper]{article}

%%%%%%%%%%%%%%%%%%%%%%%%% packages %%%%%%%%%%%%%%%%%%%%%%%%
\usepackage{amsmath}
\usepackage{amssymb}
\usepackage{amsthm}
\usepackage{amsfonts}
\usepackage{graphicx}
\usepackage[all]{xy}
\usepackage{tikz}
\usepackage{verbatim}
\usepackage[left=2cm,right=2cm,top=3cm,bottom=2.5cm]{geometry}
\usepackage{hyperref}
\usepackage{caption}
\usepackage{subcaption}
\usepackage{psfrag}

%%%%%%%%%%%%%%%%%%%%% students data %%%%%%%%%%%%%%%%%%%%%%%%
\newcommand{\student}{Akor Stanley}
\newcommand{\course}{Probability and Statistics}
\newcommand{\assignment}{1}

%%%%%%%%%%%%%%%%%%% using theorem style %%%%%%%%%%%%%%%%%%%%
\newtheorem{thm}{Theorem}
\newtheorem{lem}[thm]{Lemma}
\newtheorem{defn}[thm]{Definition}
\newtheorem{exa}[thm]{Example}
\newtheorem{rem}[thm]{Remark}
\newtheorem{coro}[thm]{Corollary}
\newtheorem{quest}{Question}[section]

%%%%%%%%%%%%%%  Shortcut for usual set of numbers  %%%%%%%%%%%

\newcommand{\N}{\mathbb{N}}
\newcommand{\Z}{\mathbb{Z}}
\newcommand{\Q}{\mathbb{Q}}
\newcommand{\R}{\mathbb{R}}
\newcommand{\C}{\mathbb{C}}

%%%%%%%%%%%%%%%%%%%%%%%%%%%%%%%%%%%%%%%%%%%%%%%%%%%%%%%555
\begin{document}

%%%%%%%%%%%%%%%%%%%%%%% title page %%%%%%%%%%%%%%%%%%%%%%%%%%
\thispagestyle{empty}
\begin{center}
\textbf{AFRICAN INSTITUTE FOR MATHEMATICAL SCIENCES \\[0.5cm]
(AIMS RWANDA, KIGALI)}
\vspace{1.0cm}
\end{center}

%%%%%%%%%%%%%%%%%%%%% assignment information %%%%%%%%%%%%%%%%
\noindent
\rule{17cm}{0.2cm}\\[0.3cm]
Name: \student \hfill Assignment Number: \assignment\\[0.1cm]
Course: \course \hfill Date: \today\\
\rule{17cm}{0.05cm}
\vspace{1.0cm}




\section*{Question 1}
\begin{itemize}
\item[(1)] The expression for the events $A,B$ happening and a third event $C$ not happening is the intercept of  $A,B$ and the intercept of $C's$ complement $i.e$ $A\cap B\cap \bar{C}$.

\item[(2)]The expression for exactly two events amongst the 3 events to occur is given as;\quad$\left( A\cap B \cap \bar{C} \right) \cup \left( A\cap \bar{B} \cap C \right) \cup \left(\bar{A} \cap \cap B \cap C\right) $ 

\item[(3)]
P(E) =$\left(A\cap B\cap \bar{C}\right)$, This can be interpreted as the Probability of $A \cap B$ without $\left(A\cap B \cap C\right)$\\
\begin{align*}
P(E)&=P\left(A \cap B \right) -P\left(A \cap B \cap C \right)\\
&=0.15-0.08\\
&=0.07
\end{align*}
\begin{align*}
P\left(F\right)&=\left( A\cap B \cap \bar{C} \right) \cup \left( A\cap \bar{B} \cap C \right) \cup \left(\bar{A} \cap B \cap C\right)\\
&=P\left(A \cap B \right) -P\left(A \cap B \cap C \right)+P\left(A \cap C \right) -P\left(A \cap B \cap C \right)+P\left(B \cap C\right) -P\left(A \cap B\cap C \right)\\
&=\left(0.15-0.08\right)+\left(\left(0.2-0.08\right)\right)+\left(0.15-0.08\right)\\
&=0.26
\end{align*}

\end{itemize}

\section*{Question 2}
\begin{itemize}

\item[(1)]
\begin{align*}
A_{1}=\{ aaa,abc,acb\}\\
A_{2}=\{ aaa, bac,cab\}\\
A_{3}=\{ aaa,bca,cba\}
\end{align*}

\item[(2)] No

\item[(3)] Yes

\end{itemize}
\section*{Question 3}
\begin{itemize}
\item[(1)] The probability of picking a defective item is:\\
$P\left(D|M_{1}\right)=0.06$,\quad $P\left(D|M_{2}\right)=0.05$,\quad$P\left(D|M_{3}\right)=0.08$,\quad$P\left(D|M_{1}\right)=0.08$,\quad $P(M_{1})=0.2$,\quad$P(M_{2})=0.2$,\quad$P(M_{3})=0.3$,\quad$P(M_{3})=0.3$.\\
\\
Applying Baye's theorem, we can obtain the probability of picking a defective item $i.e\quad P(D)$.\\
\begin{align*}
P(D)&=P\left(D|M_{1}\right)P\left(M_{1}\right)+P\left(D|M_{2}\right)P\left(M_{2}\right)+P\left(D|M_{3}\right)P\left(M_{3}\right)+P\left(D|M_{4}\right)P\left(M_{4}\right)\\
&=(0.06)(0.2)+(0.05)(0.2)+(0.08)(0.3)+(0.08)(0.3)\\
&=0.07
\end{align*}


\item[(2)]
\begin{align*}
P\left(M_{2}|D\right)&=\frac{P(M_{2}\cap D)}{P(D)}\\
&=\frac{0.2\times0.05}{0.07}\\
&=\frac{1}{7}
\end{align*}
\end{itemize}

\section*{Question 4}
Let $T_{1}=\frac{1}{3}$ represent the probability with which the first person succeeds and $R_{1}=\frac{1}{4}$ represent the probability of the second person's success. The probability of person 1 succeeding before 2 can be obtained as follows;\\
\\
Let $T$ represent the required probability, This probability  can be obtained from the countability theorem as;\\
$$P\left(T\right)=\sum _{i=0}^{\infty}P\left(T_{i}\right)$$
$P\left(T_{1}=\frac{1}{3}\right)$, probability $T_{2} =\left(\frac{2}{3}\right)\left(\frac{3}{4}\right)\left(\frac{1}{2}\right)$, this occurs when both person 1 and 2 fails and then on the second trial person 1 succeeds, $T_{3} =\left(\frac{2}{3}\right)\left(\frac{3}{4}\right)\left(\frac{1}{2}\right)\left(\frac{1}{2}\right)$, we can can obtain  $T_{4}, T_{5},.....,T_{n}$ by multipying the subsequent terms by $\frac{1}{2}$. Thus the required probability is;\\
\begin{align*}
P\left(T\right)&=\sum _{i=0} ^{\infty}\left(\frac{1}{3}\right)\left(\frac{1}{2}\right)^{n}\\
&=\left(\frac{1}{3}\right)\sum _{i=0} ^{\infty}\left(\frac{1}{2}\right)^{n} \quad  \quad\quad\quad\quad\quad\quad when  \quad n=1,\\
\end{align*}
Recall that, $\sum _{i=0}^{\infty}=\frac{1}{1-x}, \quad \quad|x|<0$, where $x=\frac{1}{2}$ in this case. We can then apply this condition to obtain our solution as;\\
\begin{align*}
P\left(T\right)&=\left(\frac{1}{3}\right)\sum _{i=0} ^{\infty}\left(\frac{1}{1-\frac{1}{2}}\right)^{1}\\
&=\frac{2}{3}
\end{align*}
\end{document}